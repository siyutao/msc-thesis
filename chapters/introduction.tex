\chapter{Introduction}\label{chapter:introduction}

Verbal valency deals with the relationship between verbs and their arguments, such as subjects and objects. It occupies a central place in investigations into language structure, as it lies at the interface between syntax and semantics, as well as between grammar and lexicon. It also exhibits considerable cross-lingual variations. A typological study of verb valency is therefore crucial to getting a better picture of the range and patterns of variations as well as to understanding the underlying causes of the similarities and differences.

However, the study of cross-lingual differences and similarities in valency systems is not without difficulties. This reflects a diversity of both theoretical and methodological challenges that result from the same central role valency plays in language structure. 

On the theoretical side, valency frames are alternately viewed either as syntactic expressions of lexical semantics, where the argument structure is determined by verbal meaning, or constructions in their own right that interact with verbs and their arguments to produce sentence meaning. The corollary of this debate extends to the relationship between valency structures and the lexicon: whereas the former view favors a lexeme-based approach, where valency information is an inherent part of a verb's lexical entry, the latter advocates a frame-based approach, where the valency frames are separate constructions with its own semantic contribution associated with but not determined by the verb.

The theoretical debates inevitably have implications for typological research on valency and, of greater interest to typology, valency systems. Most prominently, this makes agreeing on a sound \textit{tertium comparationis} (Latin for ``the third part of comparison''), or shared point of comparison, difficult. A lexeme-based approach would argue for verb classes to be the basis of comparison between languages, while a frame-based approach would attempt to find cross-lingually valid valency frames and see how they are distributed in different languages. However, regardless of the view subscribed, features of valency systems on their own pose methodological challenges. This is namely due to the fact that neither verb classes nor valency frames lend themselves to easy categorical types of how a language organize them, at least not in the same way a language can be said to have a SOV, SVO or OSV word order. 

Such theoretical and methodological complexity also translates to challenges in formulating any expected cross-lingual similarities and differences and what may have caused them. This present thesis makes an attempt to address them. The goal of the study is twofold: (1) Firstly, it advocates for a quantitative and functionalist approach to the study of verb valency. By employing empirical and corpus-based methods, this approach remains relatively theory-agnostic and enables the characterization of valency systems without relying on rigid categorical types, which in turn facilitates cross-lingual comparison; (2) Secondly, it seeks to be explanatory as well and uses the methods developed to identify typological differences across languages as well the test hypotheses for possible universals motivated by view of language as having evolved for communication and thus language structure as reflecting pressures in the communication context.

The remainder of this thesis is structured as follows: §\ref{chapter:background} provides more background and theoretical framework on functional typology, dependency grammar, and valency grammar. The main data source, Universal Dependencies treebanks, is introduced in §\ref{chapter:data}, where its suitability to typological research in general and compatibility with this study in particular are highlighted. §\ref{chapter:transitivity} presents a first experiment using corpus-based versions of the transitivity ratio metric and shows that they reveal areal and genetic patterns of transitivity among languages. §\ref{chapter:entropy} introduces a further set of experiments using novel entropy-based metrics to measure the average surprisal of valency frames given the verb and vice versa. Hypotheses regarding the correlation between these metrics and verb or valency frame frequency are posited and tested, suggesting that learnability and efficiency effects shape valency structures of languages in a universal way. Lastly, §\ref{chapter:conclusion} discusses the implications of the studies and concludes the thesis.