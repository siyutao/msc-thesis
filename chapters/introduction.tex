\chapter{Introduction}

The starting point of this study is the assumption, consistent with those behind \citet{levin1993} and other work on \textit{verb classes}, that the syntactic behavior of verbs are at least in part determined by their lexical semantics, and that, as such, verb classes based on their syntactic distribution should be semantically coherent as well. This study will test this assumption computationally by performing clustering experiments on a subset of UD treebanks in order to explore whether the UD annotations support an automated induction of the valency frames in a language and whether verb classes can be further inducted based on the distribution of verbs across the valency frames. In the process of the experiments, factors that have an impact on the outcome of clustering, particularly with respect to data quantity and quality, as well as typological features of languages, will be examined. The results of these clustering experiments will then, in combination with a computationally derived cross-lingual lexicon, support typological investigations into possible universals in the organization of verbal lexicon.


% - Typological study of verb valency (verb classes/ etc) is difficult
% - theoretical underpinning
%     - interface of syntax and semantics (disagreement)
%     - interface of grammar and lexicon (disagreement)
% - implications for typology
%     - basis for comparison hard to agree upon
%         - categorical types are difficult to arrive at
%             - within a language already
%             - even more difficult to compare (cf transitivity hierarchy)
%         - cross-lingual comparison
%     - typological similarities and differences (hypothesis)
%         - what kind of differences?
%         - what kind of universals?
% - The goal of the study is 
% - propose quantitative and functionalist approach to the study of valency
%     - able to characterize valency features / argument structure without relying on categorical types
%     - show that they are able to capture cross-lingual differences through an experiment on transitivity
%     - show that aspects of how verbal lexicon is organized reflect universal pressures on language in a communication context such as learnability and efficiency through several experiments using information theory based metrics

Universal Dependencies (UD) treebanks, a multilingual collection of dependency treebanks based on a shared, cross-lingually consistent annotation scheme \citep{nivre2020} and covering 138 languages with 243 treebanks in its most recent \texttt{v2.11} release \citep{universaldep}, have enabled significant advances in the development of multilingual dependency parsers and other NLP technologies \citep{zeman2017, zeman2018}. This proposed thesis will explore their potential in typology research through a cross-lingual quantitative study of verbal valency systems.

We develop a quantitative methodology. The methodology is described along with experiment design in the respective sections.