\chapter{Conclusion and outlook}\label{chapter:conclusion}

As indicated by the title, this thesis reports on several experiments on the cross-linguistic patterns in verbal valency systems. The exploratory nature of the experiments and the range of experiments make it easy to lose track of the overarching research questions. I summarize the experiments and their findings in this section, and discuss them in conjunction to how they help answer these questions. 

The two main research questions of the thesis are: (1) Do quantitative methods provide a way to characterize verbal valency and verbal valency systems in a manner that facilitates cross-lingual comparison? (2) If so, do the results allow us to draw inferences about cross-lingual differences as well as commonalities in how languages structure their verbal valency systems.

Two substantive experiments (1-2) are presented that address both research questions from different perspectives. Two additional experiments (3-4) point to possible future directions of work. 

Experiment 1 starts with a more basic measure for one aspect of valency, i.e. transitivity. Using corpus-based and quantitative versions of the transitivity ratio metric, it shows a general but not inviolable common tendency of languages to structure their verbal lexicon towards both ends of the transitivity spectrum, and show that the metric indicates areal and genetic patterns of transitivity between languages. Experiment 2-4 expands the scope of investigation and use information-theoretic metrics of valency frame entropy as a more holistic characterization of verbal valency. Correspondingly, linguistic differences and universals are explored with hypotheses motivated by cognitive linguistics approaches, taking learnability and efficiency constraints on language structure into account.

The results from the experiments support the view that how languages organize their verbal valency systems as well as the parts of verbal lexicon and grammar that are relevant shares cross-linguistic similarities shaped by common constraints but also shows differences in the different strategies they employ. It is relatively neutral in terms of lexeme- vs. frame-based view of valency but results from experiment 4 also suggests that the two views may not be irreconcilable with different languages leveraging different parts of the grammar to achieve the same communicative goals.

\section{Note on reproducibility}
Code and results for the thesis will be made available at the following repository: \url{https://github.com/siyutao/verbal-valency-ud}
