\chapter{Experiments and analysis}
\section{Experiment 1: Transitivity ratios}\label{sec:exp1}
\subsection{From transitivity categories to transitivity ratio}
If we were to take a top-down approach to the task of verb classification, i.e., to categorize verbs of a language into verb classes according to their syntacto-semantic properties and behavior, we could well imagine ourselves with fine-grained verb classes à la \citet{levin1993} in the end, but will likely have to start with more basic distinctions and, first among them, that of verb \textit{transitivity}. 

In contrast with subtler parameters of valency to be used in further experiments, the notion of \textit{transitivity}, i.e., whether a verb can take one or more objects, is probably familiar to any of us who has ever been in a foreign language classroom. Traditionally, a binary distinction is made between \textit{intransitive} verbs, which take only a subject and no objects and \textit{transitive} verbs, which take one or more objects. Further categories are used to deal with finer distinctions \textit{ditransitive} verbs (those taking two objects), ambitransitive verbs (those that can be used both transitively and intransitively), etc.

\todo{semantic aspects of transitivity, transitivity hierarchy}

From a functionalist perspective, however, even finer categorical distinctions do not capture the usage patterns of the verbs. A telling early corpus-based study \citep{biber1998} touches upon this when comparing the English verbs \textit{begin} and \textit{start}. At first glance, English appears to have provided us with two verbs that are not only semantically synonymous but share valency properties as well, as they can both be used in transitive and intransitive constructions:

\begin{exe}
\ex\label{example-begin_start}
  \begin{xlist}
  \ex{I had better issue a survival kit before we \textit{start}/\textit{begin}.\\ \strut\hfill \textbf{intransitive}}
  \ex{Then they \textit{started}/\textit{began} the quota system.\\ \strut\hfill \textbf{transitive with noun phrase}}
  \ex{They'd \textit{started}/\textit{begun} leaving before I arrived.\\ \strut\hfill \textbf{transitive with \textit{-ing} clause}}
  \ex{One of the wheels had \textit{started}/\textit{begun} to wobble.\\ \strut\hfill \textbf{transitive with \textit{to} clause}}
  \end{xlist}
\end{exe}

This however belies the different usage patterns exhibited by these verbs, as \citet[95]{biber1998} demonstrate with statistics from the British National Corpus (BNC): while both uses are present for both verbs, \textit{begin} is used more often in a transitive frame than \textit{start} across different genres: in fiction, 78\% of \textit{begin} occurrences (196/250) are with various transitive patterns vs. only 60\% for \textit{start} (149/250); transitive uses are less frequent in academic texts in general but the observation of relatively higher transitivity for \textit{begin} still holds (57\% vs. 36\%, or 110/192 vs. 51/142).

% construction grammar - flexibility in transitivity

The UD annotations provide us with good facility to investigate transitivity at a lexeme-level quantitatively, as it would allow us similar insights but with a wider range of languages and better coverage of their respective verbal lexicon. The relevant dependency relations \textsc{nsubj} and \textsc{obj} are defined to cover respectively the first and second core arguments of a verb with their typical syntactic roles as subject and object. This is defined without respect to specific cases (even though typically the accusative in languages with a case system) or semantic roles (even though they would typically be the proto-agent and proto-patient) in an effort to not prejudice the scheme with \textit{a priori} categories to the extent possible. The renaming of the \textsc{dobj} relation to \textsc{obj}, among the changes introduced by UD v2 \citep{nivre2020}, reflects the same effort.

Given the clear and typologically sound UD dependency relations, one can be forgiven for having already started up the code editor and tried to calculate the transitivity ratio based on them. On closer examination, however, we see that arriving at a clear definition of transitivity is in fact not trivial.

We consider four different definitions of a quantitative transitivity ratio within the UD annotation scheme:

\begin{enumerate}
    \item the number of verb instances with both \textsc{nsubj} and \textsc{obj} dependents, divided by the number of verb instances with an \textsc{nsubj} dependent
    \item the number of verb instances with an \textsc{obj} dependent, divided by the number of verb instances with an \textsc{nsubj} dependent
    \item the number of verb instances with an \textsc{obj} dependent, divided by the total number of verb instances
    \item the number of verb instances with an \textsc{obj} dependent, divided by the number of verb instances with either an \textsc{nsubj} or an \textsc{obj} dependent
\end{enumerate}

Def. 1 is an attempt at enforcing the definition of the transitive object as the \textit{second} core argument of the verb by excluding from calculation instances where the first core argument (i.e., subject) is not realized. This turns out counterproductive for two reasons. Firstly, this does not sit well with the focus on transitivity, as instances of verb use where the subject is not expressed are not per se evidence against the fact that the verb is taking a transitive object; secondly, this is undesirable from a typological perspective as well, as the metric would be biased against pro-drop languages where the pronouns would often be dropped when inferrable from grammar or context. \todo{an example sentence from a pro-drop language to illustrate}

Def. 2 comes as a reaction to the deficiencies of Def. 1, namely by dropping the requirement in the numerator for verbs to have an \textsc{nsubj} dependent. However, we are now left with the opposite problem if we are to account for typological variations, where pro-drop languages are likely to have a smaller denominator, resulting in an undesired higher transitivity ratio. The pendular biases tempt us to simply drop the \textsc{nsubj} requirement in the denominator too, arriving at Def. 3, which appears to be an elegant and simple solution. That is, until we realize the verb instances where both subject and object are dropped would affect the denominator, and such usage, e.g., non-predicative usage of verbs, is unlikely to be equally frequent in different languages and would therefore interfere with the cross-lingual comparability of our transitivity ratio. Taking all these potential drawbacks into consideration, we propose Def. 4 with the number of verb instances with either an \textsc{nsubj} or an \textsc{obj} dependent in the denominator. 

\begin{table}[ht]
    \centering
    \begin{tabularx}{0.5\textwidth}{cX}
    {\#} & \textbf{Definition} \\
    \hline
    1&$[+\textsc{nsubj}, +\textsc{obj}]/ [+\textsc{subj}]$ \\
    2&$[+\textsc{obj}] / [+\textsc{nsubj}]$  \\
    3&$[+\textsc{obj}] / [\pm\textsc{nsubj}, \pm\textsc{obj}]$\\
    4&$[+\textsc{obj}] / [+\textsc{nsubj}] \text{ or } [+\textsc{obj}]$
    \end{tabularx}
    \caption{Potential definitions of transitivity ratio considered in §\ref{sec:exp1}, represented with feature matrices}\label{tab:transitivity-defs}
\end{table}  

While we have a strong case for Def. 4 being the most principled definition, we nevertheless implement all four definitions in this experiment to empirically verify our intuitions. They are also represented with feature matrices in Tab.~\ref{tab:transitivity-defs} for quick reference.

\subsection{Transitivity ratio in the lexicon}

In the experiment, we go through all eligible UD corpora and first compile transitivity ratio statistics for each verb lexeme based on the definitions. From there, we compute the per-language statistics that will become the basis for our cross-lingual comparison: we calculate the lexeme-level and token-level transitivity ratios for each language, respectively the arithmetic mean of the lexeme transitivity ratios and the mean of lexeme transitivity ratios weighted by the frequency of the lexeme. In addition to our transitivity ratio metrics, we will also calculate an additional metric, percentage of transitive verbs, i.e., the percentage of verbs in the observed lexicon that are not strictly intransitive (defined as never observed to take an \textsc{obj}), which should correspond better with the traditional binary distinction between transitive and intransitive verbs.

We perform the experiment on the selected subset of UD data as described in §\ref{subsec:data_ud}. For the analysis, we include only languages with at least 100 observed verb lexemes (56 out of 79 languages); the full results from the experiments can be found in the accompanying data and appendices. 

To compare between the different definitions of transitivity, we compute Spearman's rank correlations between the lexeme- and token-level means of transitivity ratios according to each of our four definitions, as well as between the transitive verb percentage and each of them. The correlation statistics are listed in Tab.~\ref{tab:transitivity_spearmanr}. We observe overall strong correlations between the mean transitivity ratios at lexeme- and token-levels for all four definitions, with the highest observed for definition 4 ($\rho(54)=.87, p=.000$) and lowest observed for definition 3 ($\rho(54)=.77, p=.000$). This is not surprising as we have no reason to expect the more frequent verbs to behave differently from the less frequent verbs with regard to transitivity ratios. This can be confirmed by correlation tests between verb frequency and verb transitivity ratios for each language, which show no strong correlation (the mean absolute value of the Spearman's rank correlations for the languages is 0.087).

\begin{table}[ht]{}
    \centering
    \small
    \begin{tabularx}{\textwidth}{lXXX}
      \toprule
      def. & lexeme tr.,\newline token tr. & tr. verb \%,\newline lexeme tr. & tr. verb \%,\newline token tr. \\
      \midrule
      1 & $\rho(54)=.83, p=.000$ & $\rho(54)=.61, p=.000$ & $\rho(54)=.70, p=.000$ \\
      2 & $\rho(54)=.81, p=.000$ & $\rho(54)=-.17, p=.221$ & $\rho(54)=-.03, p=.817$ \\
      3 & $\rho(54)=.77, p=.000$ & $\rho(54)=.64, p=.000$ & $\rho(54)=.73, p=.000$ \\
      4 & $\rho(54)=.87, p=.000$ & $\rho(54)=.61, p=.000$ & $\rho(54)=.61, p=.000$ \\
      \bottomrule
    \end{tabularx}
    \caption{Spearman's rank correlation between the transitivity metrics}\label{tab:transitivity_spearmanr} 
\end{table}  

The correlation statistics between the transitive verb percentages and the transitivity ratios are slightly more revealing, if nothing else, they help us eliminate definition 2 from the competition as it shows no statistically significant correlation ($\rho(54)=-.17, p=.221$ for lexeme-level transitivity ratios and $\rho(54)=-.03, p=.817$ for token-level) while strong correlations are observed for all three other definitions (see Tab.~\ref{tab:transitivity_spearmanr}).

In the absence of a strong practical reason based on correlation statistics to select a definition among the remaining three over the others, we proceed with the rest of the analysis with results using def. 4, which we had considered \textit{a priori} to be the most principled. Furthermore, we focus on token-level transitivity ratios over lexeme-level ones as they are a more direct reflection of language use in the corpora.

\subsection{Transitivity features of the lexicon}

\begin{sidewaysfigure}[ht]
  \centering
  \includegraphics[width=\textwidth]{figures/verb_dist_by_transitivity.pdf}
  \caption{Histograms showing the binned distribution of verbs according to their transitivity ratio in different languages}
  \label{fig:verb_dist_transitivity}
\end{sidewaysfigure}

\subsection{Genetic and areal patterns in transitivity}

\begin{table}[ht]
    \centering
    \small
    \begin{subtable}[c]{\textwidth}
      \centering
      \begin{tabular}{lrr|lrr}
        \toprule
        Language & \# Verbs & Tr. verb \% & Language & \# Verbs & Tr. verb \% \\
        \midrule
        Catalan & 628 & 99.2\% & Maltese & 78 & 55.1\% \\
        Galician & 341 & 97.7\% & Hebrew & 536 & 58.4\% \\
        Urdu & 69 & 97.1\% & Hungarian & 73 & 64.4\% \\
        Hindi & 207 & 97.1\% & Russian & 2583 & 65.0\% \\
        Spanish & 948 & 96.9\% & Slovak & 284 & 66.5\% \\
        Indonesian & 330 & 96.7\% & Uyghur & 93 & 66.7\% \\
        Vietnamese & 156 & 95.5\% & Coptic & 128 & 68.0\% \\
        French & 735 & 94.8\% & Old Church Slavonic & 223 & 68.6\% \\
        Gheg & 50 & 94.0\% & Polish & 1080 & 68.8\% \\
        Danish & 212 & 93.9\% & Bambara & 50 & 70.0\% \\
        English & 773 & 93.4\% & Erzya & 87 & 70.1\% \\
        Afrikaans & 119 & 93.3\% & Latvian & 794 & 71.2\% \\
        Norwegian & 765 & 92.9\% & Gothic & 201 & 73.1\% \\
        Basque & 255 & 92.9\% & Old French & 444 & 74.1\% \\
        Ancient Greek & 1127 & 92.7\% & Faroese & 117 & 74.4\% \\
        \dots & \dots & \dots & \dots & \dots & \dots \\
        \bottomrule
      \end{tabular}
      \caption{by transitive verb percentage}
      \label{tab:most_tr_by_verb_percentage}
    \end{subtable}\\
    \begin{subtable}[c]{\textwidth}
      \centering
      \begin{tabular}{lrr|lrr}
        \toprule
        Language & \# Verbs & Token tr. & Language & \# Verbs & Token tr. \\
        \midrule
        Akkadian & 76 & 75.9\% & Scottish Gaelic & 53 & 16.2\% \\
        Catalan & 628 & 75.9\% & Irish & 108 & 30.8\% \\
        Galician & 341 & 70.5\% & Maltese & 78 & 31.7\% \\
        Afrikaans & 119 & 65.3\% & Faroese & 117 & 32.6\% \\
        Urdu & 69 & 65.2\% & Japanese & 395 & 33.4\% \\
        Gheg & 50 & 64.8\% & Hebrew & 536 & 33.5\% \\
        Vietnamese & 156 & 64.8\% & Polish & 1080 & 34.0\% \\
        Thai & 77 & 64.7\% & Uyghur & 93 & 34.1\% \\
        Classical Chinese & 1192 & 63.8\% & Erzya & 87 & 36.0\% \\
        Pomak & 244 & 62.1\% & Russian & 2583 & 37.4\% \\
        Spanish & 948 & 61.1\% & Dutch & 497 & 37.5\% \\
        Hindi & 207 & 60.8\% & Latvian & 794 & 37.8\% \\
        Chinese & 380 & 60.7\% & North Sami & 76 & 38.2\% \\
        Xibe & 74 & 58.8\% & Serbian & 214 & 38.4\% \\
        Ancient Greek & 1127 & 58.5\% & Arabic & 407 & 39.7\% \\
        \dots & \dots & \dots & \dots & \dots & \dots \\
        \bottomrule
      \end{tabular}
      \caption{by token-level transitivity ratio}
      \label{tab:most_tr_by_token_mean}
    \end{subtable}
    \caption{Most and least transitive languages by different metrics}
    \label{tab:most_tr_languages}
  \end{table}

Tab.~\ref{tab:most_tr_by_verb_percentage} and \ref{tab:most_tr_by_token_mean} list the most and least `transitive' languages in our study, respectively according to the transitive verb percentage and the token-level transitivity ratio.  

Among the most transitive languages are Romance (Catalan, Spanish, French, Galician) and Germanic (German, Norwegian, English, Danish) languages of Europe, Sinitic languages (Chinese, Classical Chinese, particularly when measured by token-level transitivity ratio), Indonesian, Hindi. On the opposite end of the spectrum are Hebrew, Irish, Japanese, as well as Baltic (Lithuanian, Latvian) and Slavic (Slovak, Russian, Polish) languages, which have the lowest transitivity ratios.

To look at any potential areal patterns in transitivity, we also map our token-level transitivity ratio results for European languages in Fig.~\ref{fig:transitivity_europe}. (A less Eurocentric study of areal patterns is unfortunately difficult for the lack of enough language samples in the UD.) We find a particularly high transitivity area in the Iberian as well as another relatively high transitivity area in the Balkan peninsular, in contrast to eastern and northern Europe with lower transitivity.

Where the languages overlap, these observations match well with those from \citet{say2014}'s survey of transitivity in European languages (as measured by the percentage of verbs that are transitive from a predetermined list), who observed high transitivity areas in western Europe except Irish and south-western Balkans, and a corresponding low transitivity area in eastern Europe.

\begin{sidewaysfigure}[ht]
  \centering
  \includegraphics[width=\textwidth]{figures/transitivity_europe.pdf}
  \caption{Mean transitivity ratios in languages of Europe}
  \label{fig:transitivity_europe}
\end{sidewaysfigure}

% Dutch appears to be a special case in several ways. \todo{case study: Dutch - divergence btwn two metrics, maybe due to ergative verbs? and diff btwn Dutch vs Afrikanns - significant grammatical differences?}

\subsection{Transitivity in the lexicon}

% diff between tr. verb -> token-level tr. ratio reflects also the fact that tr. verbs can often be used intransitively
% how "free" a language is to use tr. verb  intr./vice versa is another interesting further study 



\section{Experiment 2: Valency entropy}
\subsection{Motivation}

A valency frame is of course more complicated than transitivity.

Verb Shapiro et al. 1987; Collina et al. 2001
sentence processing is affected by semantic and syntactic attributes of the verbs 
Some fMRI study (Shetreet) suggests that number of options (in terms of subcategorization and thematic) is more correlated

encoding devices cf ValPaL

Within the UD
first the presence of the dependents in and of itself
second the word order information of the dependent 
third morphological annotation

questionable about identity

used - morphosyntactic frames

\subsection{Experiment design}


To reduce the effect of artifacts, we split each distribution randomly at a fixed ratio (e.g. $1:1$) into two, resulting in two distributions $X_1,\ldots,X_n$ and $X^{\prime}_{1},\ldots,X^{\prime}_{n}$. 

For each of the argument slot, the cross-entropy between the distributions $X_k$ and $X_k^{\prime}$ with the same image $\mathcal{X}_{k}$ is

$$
H_{cross}{(X_k,X_k^{\prime}|V=v)}=
-\sum\limits_{x\in{}\mathcal{X_{k}}}{P(x|V=v)\log_{2}{P^{\prime}(x|V=v)}}
$$

And the joint cross-entropy of  $X_1,\ldots,X_n$ and $X_1^{\prime},\ldots,X_n^{\prime}$ would be

$$
H_{joint-cross}(X_{1},\ldots,X_{n},X_{1}^{\prime},\ldots,X_{n}^{\prime}|V=v)=
-\sum\limits_{x_1\in{}\mathcal{X}_1}\cdots\sum\limits_{x_n\in{}\mathcal{X}_n}{P(x_1,\ldots,x_{n}|V=v)\log_{2}P^{\prime}(x_1,\ldots,x_n|V=v)}
$$

\subsection{Results}

relationship


\begin{sidewaysfigure}[ht]
  \centering
  \includegraphics[width=\textwidth]{figures/joint_entropy_freq.pdf}
  \caption{Scatter plots showing relationship between valency frame entropy and frequency rank of verbs, color shows number of frames associated with the verb}
  \label{fig:joint_entropy_freq}
\end{sidewaysfigure}

\section{Experiment 3: Word order and case information}
\subsection{Motivation}
\subsection{Experiment design: ablation studies}
\subsection{Results}

\section{Experiment 4: Verb entropy}
\subsection{Motivation}
\subsection{Experiment design}
\subsection{Results}

\section{Experiment 5: Verb-finalness}
\subsection{Motivation}
\subsection{Experiment design}
\subsection{Results}