\addchap{Abstract}

Issues of verbal valency have long occupied a central place in the study of language structure, as it stands at the interface between syntax and semantics, as well as between grammar and lexicon. However, the cross-lingual comparison of valency systems and their general typology have proved challenging due to both disagreements on the basis of comparison and the difficulty in arriving at categorical types. This thesis advocates a quantitative and functionalist approach to tackling this challenge. Using morphosyntactically annotated data from the Universal Dependencies (UD) treebanks, it experiments with corpus-based transitivity ratio metrics and novel entropy-based measures for valency system analysis. In doing so, it hopes to contribute to a better understanding of verb valency typology and its underlying drivers. Results from the experiments reveal areal and genetic patterns of transitivity among languages and suggest universal effects of learnability and efficiency on valency systems. 